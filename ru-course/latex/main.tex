% Deze template is gemaakt door Fons van der Plas (f.vanderplas@student.ru.nl) voor het publiek domein en mag gebruikt worden **zonder vermelding van zijn naam**.
% This template was created by Fons van der Plas (f.vanderplas@student.ru.nl) for the public domain, and may be used **without attribution**.
\documentclass{article}
\usepackage[utf8]{inputenc}     % for éô
\usepackage[english]{babel}     % for proper word breaking at line ends
\usepackage[a4paper, left=1.5in, right=1.5in, top=1.5in, bottom=1.5in]{geometry}
                                % for page size and margin settings
\usepackage{graphicx}           % for ?
\usepackage{amsmath,amssymb}    % for better equations
\usepackage{amsthm}             % for better theorem styles
\usepackage{mathtools}          % for greek math symbol formatting
\usepackage{enumitem}           % for control of 'enumerate' numbering
\usepackage{listings}           % for control of 'itemize' spacing
\usepackage{todonotes}          % for clear TODO notes
\usepackage{hyperref}           % page numbers and '\ref's become clickable


%%%%%%%%%%%%%%%%%%%%%%%%%%%%%%%%
%% SET TITLE PAGE VALUES HERE %%
%%%%%%%%%%%%%%%%%%%%%%%%%%%%%%%%
%             ||               %
%             ||               %
%             \/               %

\def\thesistitle{Convolutional Neural Networks applied to Keyword Spotting using Transfer Learning}
\def\thesissubtitle{Why Transfer learning is worth a try}
\def\thesisauthorfirst{Christoph}
\def\thesisauthorsecond{Schmidl}
\def\thesisauthorstudentnumber{s4226887}
\def\thesisauthoremail{c.schmidl@student.ru.nl}
\def\thesissupervisorfirst{dr. L.F.M. }
\def\thesissupervisorsecond{ten Bosch}
\def\thesissecondreaderfirst{prof. dr. Louie}
\def\thesissecondreadersecond{Duck}
\def\thesisdate{\today}


%             /\               %
%             ||               %
%             ||               %
%%%%%%%%%%%%%%%%%%%%%%%%%%%%%%%%
%% SET TITLE PAGE VALUES HERE %%
%%%%%%%%%%%%%%%%%%%%%%%%%%%%%%%%


%% FOR PDF METADATA
\title{\thesistitle}
\author{\thesisauthorfirst\space\thesisauthorsecond}
\date{\thesisdate}

%% TODO PACKAGE
\newcommand{\towrite}[1]{\todo[inline,color=yellow!10]{TO WRITE: #1}}

%% THEOREM STYLES
\newtheorem{theorem}{Theorem}[section]
\newtheorem{corollary}{Corollary}[theorem]
\newtheorem{lemma}[theorem]{Lemma}
\newtheorem{proposition}[theorem]{Proposition}

\theoremstyle{definition}
\newtheorem{definition}[theorem]{Definition}

\theoremstyle{remark}
\newtheorem*{remark}{Remark}



%% MATH OPERATORS
\DeclareMathOperator{\supersine}{supersin}
\DeclareMathOperator{\supercosine}{supercos}

%%%%%%%%%%%%%%%%%%%%%%%

\begin{document}
\begin{titlepage}
	\thispagestyle{empty}
	\newcommand{\HRule}{\rule{\linewidth}{0.5mm}}
	\center
	\textsc{\Large Radboud University Nijmegen}\\[.7cm]
	\includegraphics[width=25mm]{img/in_dei_nomine_feliciter.eps}\\[.5cm]
	\textsc{Faculty of Science}\\[0.5cm]
	
	\HRule \\[0.4cm]
	{ \huge  \thesistitle}\\[0.1cm]
	%\textsc{\thesissubtitle}\\
	\HRule \\[.5cm]
	

	\textsc{\large Thesis in Automatic Speech Recognition (LET-REMA-LCEX10)}\\[.5cm]

% https://tex.stackexchange.com/questions/81955/align-text-in-minipage-at-same-height
	\begin{minipage}[t]{0.4\textwidth}
	\begin{flushleft} \large
	\emph{Author:}\\
	\vspace{1em}
	\thesisauthorfirst\space \textsc{\thesisauthorsecond}\\
	\thesisauthorstudentnumber\\
	\thesisauthoremail\space 
	\end{flushleft}
	\end{minipage}
	~
	\begin{minipage}[t]{0.4\textwidth}
	\begin{flushright} \large
	\emph{Supervisor:} \\
	\vspace{1em}
	\thesissupervisorfirst\space \textsc{\thesissupervisorsecond} \\[1em]
	%\emph{Second reader:} \\
	%\thesissecondreaderfirst\space \textsc{\thesissecondreadersecond}
	\end{flushright}
	\end{minipage}\\[4cm]
	\vfill
	{\large \thesisdate}\\
	\clearpage
\end{titlepage}

\tableofcontents

\newpage

\section{Introduction}

The task of keyword spotting (KWS) is interesting to different domains where a hands-free interaction experience is required or desired like Google's feature of interacting with mobile devices (include "OK Google" reference). \\

Different approaches to keyword spotting like:

\begin{itemize}
	\item Deep Neural Networks (DNNs)
	\item Convolutional Neural Networks (CNNs)
	\item (Keyword/Filler) Hiddem Markov Models (HMMs)
\end{itemize}



\begin{enumerate}
	\item Problem
	\item Background (literature overview)
	\item Research Question, Hypotheses, intro to experiment
\end{enumerate}

\subsection{Literature review}

This section contains the most prominent approaches to the KWS task which have been successfully applied in the past and serve as baseline models or inspirations for the proposed model in this thesis. 

\subsubsection{Speech Recognition: Keyword Spotting Through Image Recognition.}

\begin{itemize}
	\item Speech Recognition: Keyword Spotting Through Image Recognition. \cite{gouda2018speech}
\end{itemize}	

\textcolor{red}{Include summary about the approach of converting the long, one dimensional vector of audio data into a spectrograms and therefore making it a image classification problem.}

\subsubsection{Convolutional neural networks for small-footprint keyword spotting}

\begin{itemize}
	\item Convolutional neural networks for small-footprint keyword spotting \cite{sainath2015convolutional}
\end{itemize}	

\textcolor{red}{Include summary about the different CNNs approaches which have been put into the 3 module framework of the below framework where the DNN has been exchanged for a CNN. How do the authors handle the long, one dimensional vector?}

\subsubsection{Small-footprint keyword spotting using deep neural networks}

\begin{itemize}
	\item Small-footprint keyword spotting using deep neural networks \cite{chen2014small}
\end{itemize}	

\textcolor{red}{Include summary about the comparison between DNNs and HMMs and the general 3 module approach here: 1. Feature extraction. 2. Deep Neural Network 3. Posterior Handling. DNNs do not need a decoding algorithm like HMMs with Viterbi which makes it low latency.}

\subsubsection{Speech Commands: A Dataset for Limited-Vocabulary Speech Recognition}


\begin{itemize}
	\item Speech Commands: A Dataset for Limited-Vocabulary Speech Recognition \cite{warden2018speech}
\end{itemize}	

\textcolor{red}{Include summary and say why the Speech Commands Dataset is a good fit for this thesis. You probably do not need a Voice-activity detection (VAD) system here.}


\begin{itemize}
	\item Convolutional recurrent neural networks for small-footprint keyword spotting \cite{arik2017convolutional}
	\item Honk: A PyTorch reimplementation of convolutional neural networks for keyword spotting \cite{tang2017honk}
	\item An experimental analysis of the power consumption of convolutional neural networks for keyword spotting \cite{tang2018experimental}
	\item Transfer learning for speech recognition on a budget \cite{kunze2017transfer}
	\item Learning and transferring mid-level image representations using convolutional neural networks \cite{oquab2014learning}
	\item Deep residual learning for small-footprint keyword spotting \cite{tang2018deep}
\end{itemize}




\section{Method}

\begin{enumerate}
	\item methodology, types of analyses, selection of the method
\end{enumerate}

\section{Set-up}

\begin{enumerate}
	\item selection of the speech data, description of the data, tuning/adaptation model parameters
	\item types of experiments (generalizations to which unseen conditions, etc. )
\end{enumerate}


\section{Experiments}



\section{Analysis and Results}


\section{Discussion}

\section{Conclusion}



\section{References}




\section{Appendix}

\begin{figure}[h]
    \centering
    \includegraphics[width=1\textwidth]{img/grading.png}
    \caption{Weighted grading}
    \label{fig:my_label}
\end{figure}

\begin{itemize}
	\item the experiment(s) may be carried out in collaboration with others. In that case: specify in the “author’s statement” everybody’s contribution
	\item the thesis itself is written individually and assessed individually
	\item the ASR performance itself is not relevant for the assessment of the thesis
	\item the RQ, the literature embedding of the RQ, the description of the method, the justification and set-up of the experiment are relevant for the assessment 
	\item the general university guidelines apply (e.g., with respect to plagiarism)
	\item there is no minimum number of pages for the thesis
\end{itemize}



\section{Complex stuff}
\subsection{Domains}
Let's start with the following definition:
\begin{definition}\label{def:domain}
A set $U \subseteq \mathbb{C}$ is a \emph{domain} if:
\begin{itemize}
    \item $U$ is open in $\mathbb{C}$, and
    \item $U$ is connected.
\end{itemize}
\end{definition}


\subsection{Yumyumyumyum}
\towrite{an introduction and some examples}

\begin{theorem}[]
Suppose $n \in \mathbb{Z}$, then the following are equivalent:
\begin{enumerate}[label=\roman*.]
    \item $n > 5$.
    \item $5 > 5$.\todo{This doesn't seem right...}
    \item For each $n \in n$, we have:
    \begin{align}\label{eq:truth}
        n > n+1 > n+1^2 > \dots > n+7.
    \end{align}
    where $7$ is an arbitrary element of
    \begin{align*}
        \oint_{a}^{b} \supersine \alpha + i \supercosine \beta  db(a).
    \end{align*}
\end{enumerate}
\end{theorem}

\begin{remark}
Interesting!
\end{remark}
\begin{proof}
See \cite{Rynne2008LinearAnalysis}.
\end{proof}

\begin{figure}[h]
    \centering
    \includegraphics[width=.3\textwidth]{img/in_dei_nomine_feliciter.eps}
    \caption{Motivational illustration. Similar to \cite{Oort1958,Reed1960}.}
    \label{fig:logo}
\end{figure}

\begin{corollary}
Suppose $U \subseteq \mathbb{C}$ is a domain (see Definition \ref{def:domain}), and $f: \overline{U} \rightarrow \mathbb{C}$ is continuous on $\overline{U}$ and holomorphic on $U$. If $z \mapsto |f(z)|$ is constant on $\partial U$, then $f$ has a zero in $U$.
\end{corollary}
\begin{proof}
If not, consider $\frac{1}{f}$.
\end{proof}
The proof of this theorem is illustrated in Figure \ref{fig:logo}.



\newpage

% You can choose a citation style, 'plain' is the default
% See:
% https://www.overleaf.com/learn/latex/Bibtex_bibliography_styles

\bibliographystyle{plain}
\bibliography{references.bib}

\end{document}

% Have fun!
% -fons

% http://www2.washjeff.edu/users/rhigginbottom/latex/resources/symbols.pdf